\section{Gaußsches Eliminationsverfahren}
Das gaußsche Eliminationsverfahren ist ein Algorithmus zum Lösen von linearen Gleichungssystemen.

\begin{procedure}
Folgendes Vorgehen beschreibt das Lösen von linearen Gleichungssystemen mit dem gaußschen Eliminationsverfahren:
\begin{enumerate}
  \item Umwandeln in Matrizen-Form
  \item Unter der Hauptdiagonale die Koeffizienten Null setzen (Dreiecksform)
  \begin{enumerate}
    \item Vertauschen der Zeilen
    \item Multiplizieren einer Zeile mit einer Anderen
    \item Addieren einer Zeile mit einer Anderen
  \end{enumerate}
  \item Auflösen der Gleichungen von unten nach oben
\end{enumerate}
\end{procedure}

\begin{example}
  Das folgende Gleichungssystem soll gelöst werden.
  \begin{equation*}
    \begin{array}{rcrcrcr r}
      2x &+& y &-& z &=& 8 & L_1\\
      -3x &-& y &+& 2z &=& -11 & L_2 \\
      -2x &+& y &+& 2z &=& -3 & L_3
    \end{array}
  \end{equation*}
\begin{solution}
  \begin{equation*}
    \text{Matrix:}\left[\begin{array}{rrr|r}
      2 & 1 & -1 & 8 \\
      -3 & -1 & 2 & -11 \\
      -2 & 1 & 2 & -3
    \end{array}\right]\quad\text{Zeilenoperationen:}
    \begin{array}{rcrcr}
      L_2 &+& \frac{3}{2}L_1 &\to& L_2 \\
      L_3 &+& L_1 &\to& L_3
    \end{array}
  \end{equation*}
  \begin{equation*}
    \text{Matrix:}\left[\begin{array}{rrr|r}
      2 & 1 & -1 & 8 \\
      0 & \frac{1}{2} & \frac{1}{2} & 1 \\
      0 & 2 & 1 & 5
    \end{array}\right]\quad\text{Zeilenoperationen:}
    \begin{array}{rcrcr}
      L_3 &+& -4L_2 &\to& L_3
    \end{array}
  \end{equation*}
  \begin{equation*}
    \text{Matrix:}\left[\begin{array}{rrr|r}
      2 & 1 & -1 & 8 \\
      0 & \frac{1}{2} & \frac{1}{2} & 1 \\
      0 & 0 & -1 & 1
    \end{array}\right]\quad\text{Ergibt:}
    \begin{array}{rcr}
      x &=& 2\\
      y &=& 3\\
      z &=& -1
    \end{array}
  \end{equation*}
\end{solution}
\end{example}
